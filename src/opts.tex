\section{Optimizations}

\subsection{Possible optimizations}

% FIXME:
% Optimizations section: big table would be nice. Especially because you don't have a benchmarking result, seeing that most optimizations are still preserved (or more optimizations are added) by TB (vs. SB) would be nice
%- slide 20: put a clear sign that you're transitiioning to a different part of talk. can you visualize the protector stuff somehow?


\begin{frame}[fragile, t]
    \frametitle{{\cmark} Delay protected writes}
    \begin{onlyenv}<1-5>
        \begin{block}{}
            \begin{lstlisting}[language=rust, escapechar=@]
@\visible<2>{>}@fn write(x: &mut u64) {
@\visible<3>{>}@    *x = 42; // move down ?
@\visible<4,5>{>}@    opaque();

@ @}
            \end{lstlisting}
        \end{block}
    \end{onlyenv}
    \begin{onlyenv}<6->
        \begin{block}{}
            \begin{lstlisting}[language=rust, escapechar=@]
@ @fn write(x: &mut u64) {

@ @    opaque();
@ @    *x = 42;
@ @}
            \end{lstlisting}
        \end{block}
    \end{onlyenv}

    \begin{onlyenv}<1-5>
        \begin{block}{}
            \begin{tikzpicture}[
                every node/.append style = {anchor = west},
                grow via three points={one child at (0.3,-0.5) and two children at (0.3,-0.5) and (0.3,-1.0)},
                edge from parent path={(\tikzparentnode\tikzparentanchor) |- (\tikzchildnode\tikzchildanchor)}
                ]
                \node (nparent) at (0,0) {\texttt{?}}
                    child {node (nx) {\texttt{x:}}};

                \node<2>[right = 0cm of nx, anchor=west] {\texttt{Reserved}};
                \node<3>[right = 0cm of nx, anchor=west] {\texttt{Active}};
                \node<4>[right = 0cm of nx, anchor=west] {\texttt{Active} | \texttt{Frozen} | \texttt{Disabled}};
                \node<5>[right = 0cm of nx, anchor=west] {\texttt{Active} | \cancel{\texttt{Frozen}} | \cancel{\texttt{Disabled}}};

                \node (vert) at (7,0 |- nparent) {};
                \node<3>[anchor=west] at (vert |- nx) {\(\gets\) write};
                \node<4>[anchor=west] at (vert |- nparent) {\(\gets\) ?};
            \end{tikzpicture}
        \end{block}
    \end{onlyenv}
\end{frame}

% FIXME:
% - slide 21: "speculative read" should be slide title?


\begin{frame}[fragile, t]
    \frametitle{{\cmark} Spurious read}
    \begin{onlyenv}<1-2>
        \begin{block}{}
            \begin{lstlisting}[language=rust]
fn sum_while(incr: &u64) -> u64 {
    let mut sum = 0;

    while condition() {
        sum += *incr; // hoist ?
    }
    sum
}
            \end{lstlisting}
        \end{block}
    \end{onlyenv}

    \begin{onlyenv}<3>
        \begin{block}{}
            \begin{lstlisting}[language=rust]
fn sum_while(incr: &u64) -> u64 {
    let mut sum = 0;
    let incr = *incr;
    while condition() {
        sum += incr;
    }
    sum
}
            \end{lstlisting}
        \end{block}
    \end{onlyenv}

    \begin{onlyenv}<2->
        \begin{block}{}
            \begin{itemize}
                \item \texttt{incr} cannot be mutated
                \item thanks to the read on reborrow and the protector, \texttt{incr} is known to be readable...
                    \begin{itemize}
                        \item ...even if \texttt{condition()} is never true
                        \item ...even if \texttt{condition()} does not terminate
                    \end{itemize}
            \end{itemize}
        \end{block}
    \end{onlyenv}
\end{frame}

\subsection{Lost optimizations}

% FIXME:
% - slide 22: is stacked borrows strictly stronger than tree borrows? maybe give examples of these optimizations before? (speculative write, etc. )
% - slide 23: I didn't get this. Very hard to follow. Maybe example with intermediate states would help?


\begin{frame}
    \frametitle{Allowed by Stacked Borrows but not Tree Borrows}
    \begin{itemize}
        \item speculative write
        \item stronger read-write reorderings
    \end{itemize}

    \begin{onlyenv}<2>
        \begin{block}{}
            What would be the cost of enabling those optimizations ? \\
            (SB has no choice, TB does)
        \end{block}
    \end{onlyenv}
\end{frame}

\begin{frame}[fragile, t]
    \frametitle{Speculative writes}

    \begin{exampleblock}{Variant}
        Write to mutable references on function entry.
    \end{exampleblock}

    \begin{block}{{\cmark} Speculative writes}
        \begin{itemize}
            \item guaranteed writeable by the write on function entry
            \item still \texttt{Active} thanks to the protector
        \end{itemize}
    \end{block}

    \begin{block}{{\xmark} ``\texttt{as\_mut\_ptr}'' pattern}
        \begin{lstlisting}[language=rust, basicstyle=\ttfamily\scriptsize]
BCryptGenRandom(
    handle,
    buffer.as_mut_ptr(),   // takes `&mut buffer` despite being read-only
    buffer.len() as ULONG, // invalidates mutable borrows of `buffer`
    opts,
)
        \end{lstlisting}
    \end{block}
\end{frame}

% FIXME:
% - slide 24: the change in the opt is hard to see because stuff is shifting.


\begin{frame}[fragile, t]
    \frametitle{Read-Write reorderings}

    \begin{exampleblock}{Variant}
        On a foreign read, \texttt{Active} becomes \texttt{Disabled} instead of \texttt{Frozen}.
    \end{exampleblock}

%%% possible

    \begin{onlyenv}<2>
        \begin{block}{{\cmark} Delay arbitrary writes towards reads}
            \begin{lstlisting}[language=rust, basicstyle=\ttfamily\scriptsize]
let x = &mut *y;
*x = 42; // move down ?
opaque();

let z = *x;
            \end{lstlisting}
        \end{block}
    \end{onlyenv}

    \begin{onlyenv}<3-4>
        \begin{block}{{\cmark} Delay arbitrary writes towards reads}
            \begin{lstlisting}[language=rust, basicstyle=\ttfamily\scriptsize]
let x = &mut *y;

opaque();
*x = 42;
let z = *x;
            \end{lstlisting}
        \end{block}
    \end{onlyenv}

    \begin{onlyenv}<4>
        \begin{block}{}
            Was already possible for protected references,
            this makes it available always.
        \end{block}
    \end{onlyenv}

%%% impossible

    \begin{onlyenv}<5>
        \begin{block}{{\xmark} Arbitrary read reordering}
            \begin{lstlisting}[language=rust]
let x = &mut *z;
*x = 42;
let y = *x; // move down ?
let w = *z;

            \end{lstlisting}
        \end{block}
    \end{onlyenv}

    \begin{onlyenv}<6-7>
        \begin{block}{{\xmark} Arbitrary read reordering}
            \begin{lstlisting}[language=rust]
let x = &mut *z;
*x = 42;

let w = *z; // invalidates `x`
let y = *x; // UB!
            \end{lstlisting}
        \end{block}
    \end{onlyenv}

    \begin{onlyenv}<7>
        \begin{block}{}
            More UB is not \textit{always} more optimizations!\\
            Same phenomenon with making non-two-phase directly \texttt{Active}.
        \end{block}
    \end{onlyenv}
\end{frame}

\begin{frame}
    \frametitle{Summary}
    \begin{itemize}
        \item read reorderings, spurious reads are possible
        \item spurious writes and unprotected write reorderings are not
        \item the model can be strengthened to justify stronger optimizations for
            writes, but at the cost of commonly written patterns and some read-read reorderings
    \end{itemize}
\end{frame}
