\section{Optimizations}

\begin{frame}
    \frametitle{Some standard optimizations}
    \newcommand{\asterisk}{\textsuperscript{*}}
    \newcommand{\noasterisk}{\phantom{\asterisk}}
    \begin{tabular}{|l|c|c|l}
        \cline{1-3}
        Possible in...                     & SB     & TB &\\
        \cline{1-3}
        Swap call-read \(\to\) read-call (speculative)
            & \cmark\noasterisk
            & \cmark\noasterisk
            &\\
        \cline{1-3}
        Swap read-call \(\to\) call-read
            & \cmark\asterisk
            & \cmark\asterisk
            &\\
        \(\triangleright\)\scriptsize~ Swap read-read' \(\to\) read'-read
            & \scriptsize \cmark\asterisk
            & \scriptsize \cmark\noasterisk
            & \visible<3>{\(\gets\) TB only}\\
        \cline{1-3}
        Swap call-write \(\to\) write-call (speculative)
            & \cmark\asterisk
            & \xmark\noasterisk
            & \visible<2>{\(\gets\) SB only}\\
        \(\triangleright\)\scriptsize~ Swap write-call-write \(\to\) write-write-call
            & \scriptsize \cmark\asterisk
            & \scriptsize \cmark\asterisk
            &\\
        \cline{1-3}
        Swap write-call \(\to\) call-write
            & \cmark\asterisk
            & \cmark\asterisk
            &\\
        \(\triangleright\)\scriptsize~ Swap write-write'-read \(\to\) write'-write-read
            & \scriptsize \cmark\noasterisk
            & \scriptsize \cmark\asterisk
            & \visible<2>{\(\gets\) SB only}\\
        \cline{1-3}
    \end{tabular}~\\~\\

    \asterisk: only for protected references
\end{frame}

\subsection{Possible optimizations: intuition}

\newcommand{\assumepath}[2]{\textover{\visible<#1>{\color{magenta}{#2}}}{}}

\begin{frame}[fragile, t]
    \frametitle{Insert speculative read}
    \begin{onlyenv}<1-4>
        \begin{block}{{\cmark} Base model}
            \begin{lstlisting}[language=rust, escapechar=@]
fn read(x: &u64) -> u64 {

    opaque(/* contains foreign access ? @\assumepath{2}{if none}\assumepath{3}{if read}\assumepath{4}{if write}\phantom{........}@ */);
    *x // (optimization: move up ?)
}
            \end{lstlisting}
        \end{block}
        \includegraphics<1>{blank.base.pdf}
        \includegraphics<2>{path.prot.shr+cr.pdf}
        \includegraphics<3>{path.prot.shr+cr+fr.pdf}
        \includegraphics<4>{path.prot.shr+fw.pdf}
    \end{onlyenv}

    \begin{onlyenv}<5>
        \begin{block}{{\cmark} Base model}
            \begin{lstlisting}[language=rust, escapechar=@]
fn read(x: &u64) -> u64 {
    let val = *x;
    opaque(/* assume no foreign write */);
    val
}
            \end{lstlisting}
        \end{block}
        \includegraphics{path.prot.shr+cr+fr.pdf}
    \end{onlyenv}
\end{frame}

\subsection{Impossible optimizations}

\begin{frame}[fragile, t]
    \frametitle{Insert speculative write}

    \begin{block}{{\xmark} Base model}
        \begin{lstlisting}[language=rust, escapechar=@]
fn foo(x: &mut u64) {

    opaque(/* contains foreign access ? @\assumepath{2}{if write}\assumepath{3}{if read}\assumepath{4}{if read+loop}\phantom{............}@ */);
    *x = 42; // (optimization: move up ?)
}
        \end{lstlisting}
    \end{block}
    \includegraphics<1>{blank.base.pdf}
    \includegraphics<2>{path.prot.mut+fw.pdf}
    \includegraphics<3>{path.prot.mut+fr+cw.pdf}
    \includegraphics<4>{path.prot.mut+fr.pdf}
\end{frame}

\begin{frame}[fragile, t]
    \frametitle{Insert speculative write: blocker}

    \begin{exampleblock}{\texttt{as\_mut\_ptr}: \textover{\visible<1-3>{base model}}{}\textover{\visible<4-5>{strengthened with speculative writes}}{}}
        \begin{itemize}
            \item \texttt{\&mut [T] -> *mut T}
            \item returns \textover{\visible<1-3>{a \texttt{Reserved}}}{}\textover{\visible<4-5>{an \texttt{Active}}}{}\phantom{aaaaaaaaaa.} child of the input
        \end{itemize}
    \end{exampleblock}

    \begin{block}{\textover{\visible<1-3>{\cmark}}{}\textover{\visible<4-5>{\xmark}}{}\phantom{\xmark} Common pattern}
        \begin{lstlisting}[language=rust, basicstyle=\ttfamily\scriptsize]
let raw = buf.as_mut_ptr();
let shr = buf.as_ptr().add(1);
copy_nonoverlapping(shr, raw, 1);
        \end{lstlisting}
    \end{block}

    \begin{onlyenv}<1-3>
        \begin{onlyenv}<1>
            \begin{tikzpicture}[
                every node/.append style = {anchor = west},
                grow via three points={one child at (0.3,-0.5) and two children at (0.3,-0.5) and (0.3,-1.0)},
                edge from parent path={(\tikzparentnode\tikzparentanchor) |- (\tikzchildnode\tikzchildanchor)}
                ]
                \node[anchor=west] (nbuf) at (0,0) {\texttt{buf:\textover{ Active}{}}}
                        child {node {\texttt{...}}
                            child {node (nraw) {\texttt{raw: Reserved}}}}
                        ;
            \end{tikzpicture}
        \end{onlyenv}
        \begin{onlyenv}<2>
            \begin{tikzpicture}[
                every node/.append style = {anchor = west},
                grow via three points={one child at (0.3,-0.5) and two children at (0.3,-0.5) and (0.3,-1.0)},
                edge from parent path={(\tikzparentnode\tikzparentanchor) |- (\tikzchildnode\tikzchildanchor)}
                ]
                \node[anchor=west] (nbuf) at (0,0) {\texttt{buf:\textover{ Active}{}}}
                        child {node {\texttt{...}}
                            child {node (nraw) {\texttt{raw: Reserved}}}}
                        child[missing] {}
                        child {node {\texttt{...}}
                            child {node (nshr) {\texttt{shr: Frozen}}}}
                        ;
            \end{tikzpicture}
        \end{onlyenv}
        \begin{onlyenv}<3>
            \begin{tikzpicture}[
                every node/.append style = {anchor = west},
                grow via three points={one child at (0.3,-0.5) and two children at (0.3,-0.5) and (0.3,-1.0)},
                edge from parent path={(\tikzparentnode\tikzparentanchor) |- (\tikzchildnode\tikzchildanchor)}
                ]
                \node[anchor=west] (nbuf) at (0,0) {\texttt{buf:\textover{ Active}{}}}
                        child {node {\texttt{...}}
                            child {node (nraw) {\texttt{raw: Active}}}}
                        child[missing] {}
                        child {node {\texttt{...}}
                            child {node (nshr) {\texttt{shr: Disabled}}}}
                        ;
            \end{tikzpicture}
        \end{onlyenv}

    \end{onlyenv}

    \begin{onlyenv}<4-5>
        \begin{onlyenv}<4>
            \begin{tikzpicture}[
                every node/.append style = {anchor = west},
                grow via three points={one child at (0.3,-0.5) and two children at (0.3,-0.5) and (0.3,-1.0)},
                edge from parent path={(\tikzparentnode\tikzparentanchor) |- (\tikzchildnode\tikzchildanchor)}
                ]
                \node[anchor=west] (nbuf) at (0,0) {\texttt{buf:\textover{ Active}{}}}
                        child {node {\texttt{...}}
                            child {node (nraw) {\texttt{raw: Active}}}}
                        ;
            \end{tikzpicture}
        \end{onlyenv}
        \begin{onlyenv}<5>
            \begin{tikzpicture}[
                every node/.append style = {anchor = west},
                grow via three points={one child at (0.3,-0.5) and two children at (0.3,-0.5) and (0.3,-1.0)},
                edge from parent path={(\tikzparentnode\tikzparentanchor) |- (\tikzchildnode\tikzchildanchor)}
                ]
                \node[anchor=west] (nbuf) at (0,0) {\texttt{buf:\textover{ Active}{}}}
                        child {node {\texttt{...}}
                            child {node (nraw) {\texttt{raw: Frozen}}}}
                        child[missing] {}
                        child {node {\texttt{...}}
                            child {node (nshr) {\texttt{shr: Frozen}}}}
                        ;
            \end{tikzpicture}
        \end{onlyenv}
    \end{onlyenv}
\end{frame}

\subsection*{}

\begin{frame}
    \frametitle{Summary}
    \begin{itemize}
        \item TB allows read reorderings (SB does not)
        \item TB allows speculative reads (SB as well)
        \item TB forbids speculative writes (SB allows them)
            \begin{itemize}
                \item the model can be strengthened to justify these optimizations...
                \item ...at the cost of common patterns.
            \end{itemize}
    \end{itemize}
\end{frame}
