\section{Deriving rules}

\begin{frame}
    \frametitle{How many permissions ?}
    In short: one permission per ``kind of pointer''
    \begin{itemize}
        \item (interior) mutability,
        \item lifetime information,
        \item creation context,
        \item ...
    \end{itemize}~\\

    Guarantees required of pointers determine behavior of permissions:
    \begin{itemize}
        \item pointer allows mutation \\
            \(\qquad\Rightarrow\) permission allows child writes
        \item pointer guarantees uniqueness\\
            \(\qquad\Rightarrow\) permission is invalidated by foreign accesses
        \item ...
    \end{itemize}
\end{frame}

\begin{frame}[t]
    \frametitle{Reserved, Active, Frozen, Disabled}
    Basic permissions to represent
    \begin{itemize}
        \item two phase borrowed (mutable in the future): \texttt{Reserved},
        \item unique mutable references: \texttt{Active},
        \item shared immutable references: \texttt{Frozen},
        \item lifetime ended: \texttt{Disabled}.
    \end{itemize}

    \begin{onlyenv}<1>
        \begin{block}{Child \textover{read}{write}: must allow reading}
        \includegraphics{steps.base.cr.pdf}
        \end{block}
    \end{onlyenv}

    \begin{onlyenv}<2>
        \begin{block}{Child write: must allow writing}
        \includegraphics{steps.base.cr+cw.pdf}
        \end{block}
    \end{onlyenv}

    \begin{onlyenv}<3>
        \begin{block}{Foreign \textover{read}{write}: no longer unique}
        \includegraphics{steps.base.cr+cw+fr.pdf}
        \end{block}
    \end{onlyenv}

    \begin{onlyenv}<4>
        \begin{block}{Foreign write: no longer immutable}
        \includegraphics{steps.base.cr+cw+fr+fw.pdf}
        \end{block}
    \end{onlyenv}
\end{frame}

%\begin{frame}[fragile, t]
%    \frametitle{Example: \texttt{Reserved} in action}
%    \begin{alertblock}{Two-phase borrows}
%        Mutable reborrows in function arguments tolerate shared reborrows
%        until function entry.
%    \end{alertblock}
%
%    \begin{block}{}
%        \begin{minipage}{0.4\textwidth}
%            \begin{lstlisting}[language=rust, escapechar=\@]
%@              @fn main() {
%@\visible<2>{\color{red}>}@    let mut v =
%@\visible<2>{\color{red}>}@        vec![1usize];
%@\visible<3,5>{\color{red}>}@    v.push(
%@\visible<4,5>{\color{red}>}@        v.len()
%@\visible<5>{\color{red}>}@    );
%@              @}
%            \end{lstlisting}
%        \end{minipage}
%        \vline
%        \begin{minipage}{0.40\textwidth}
%            \begin{tikzpicture}[
%                every node/.append style = {anchor = west},
%                grow via three points={one child at (0.3,-0.5) and two children at (0.3,-0.5) and (0.3,-1.0)},
%                edge from parent path={(\tikzparentnode\tikzparentanchor) |- (\tikzchildnode\tikzchildanchor)}
%                ]
%                \node (nv) at (0,0) {\texttt{v:} {\visible<2->{\textover{\texttt{Active}}{}}}}
%                        child {node (nvpush) {\texttt{v\textsubscript{push}:} {\visible<3,4>{\textover{\texttt{Reserved}}{}}}{\visible<5>{\textover{\texttt{Active}}{}}}}}
%                        child {node (nvlen) {\texttt{v\textsubscript{len}:} {\visible<4>{\textover{\texttt{Frozen}}{}}}{\visible<5>{\textover{\texttt{Disabled}}{}}}}}
%                        ;
%
%                \node (vert) at (4,0 |- nv) {};
%                \node at (vert |- nv) {{\visible<3,4>{\(\gets\) reborrow}}};
%                \node<4> at (vert |- nvlen) {\(\gets\) read};
%                \node<5> at (vert |- nvpush) {\(\gets\) write};
%            \end{tikzpicture}~\\~\\
%            \includegraphics<1-2>[width=1.4\textwidth]{mod.base.pdf}
%            \includegraphics<3>[width=1.4\textwidth]{path.base.mut.pdf}
%            \includegraphics<4>[width=1.4\textwidth]{path.base.mut+fr.pdf}
%            \includegraphics<5>[width=1.4\textwidth]{path.base.mut+fr+cw.pdf}
%            {}
%        \end{minipage}
%    \end{block}
%\end{frame}

\subsection{Justifying \texttt{noalias}}

\begin{frame}[fragile, t]
    \frametitle{Loss of permissions too early}
    \begin{alertblock}{LLVM \texttt{noalias} (in TB terms)}
        No foreign access during the same function call as a child access
        if at least one is a write.
    \end{alertblock}
    
    \begin{onlyenv}<1>
        \begin{center}
        \begin{tabular}{|c|c|c|c|c|}
            \hline
            1 \(\backslash\) 2 & \(\uparrow\)R & \(\uparrow\)W & \(\downarrow\)R & \(\downarrow\)W \\
            \hline
            \(\uparrow\)R &  &  &  & \(\times\) \\
            \hline
            \(\uparrow\)W &  &  & \(\times\) & \(\times\) \\
            \hline
            \(\downarrow\)R &  & \(\times\) &  &  \\
            \hline
            \(\downarrow\)W & \(\times\) & \(\times\) &  &  \\
            \hline
        \end{tabular}
        \end{center}
    \end{onlyenv}

    \begin{onlyenv}<2>
        \begin{block}{Previous model: unsound}
            \begin{lstlisting}[language=rust, escapechar=@]
fn write(x: &mut u64) {
    *x = 42; // activation
    opaque(/* foreign read for x: noalias violation */);
}
            \end{lstlisting}
        \end{block}
        \includegraphics[width=0.7\textwidth]{path.base.mut+cw+fr.pdf}
    \end{onlyenv}
\end{frame}

\begin{frame}[fragile, t]
    \frametitle{Protectors lock permissions}
    \begin{onlyenv}<1>
        \begin{exampleblock}{Intuition}
            \texttt{noalias} requires exclusive access during the entire
            function call, so we remember the set of all functions that have not yet
            returned and enforce exclusivity for their arguments.
        \end{exampleblock}

        \begin{center}
            \begin{tabular}{|c|c|c|c|c|}
                \hline
                1 \(\backslash\) 2 & \(\uparrow\)R & \(\uparrow\)W & \(\downarrow\)R & \(\downarrow\)W \\
                \hline
                \(\uparrow\)R &  &  &  & \(\times\) \\
                \hline
                \(\uparrow\)W &  &  & \(\approx\) & \(\approx\) \\
                \hline
                \(\downarrow\)R &  & \(\times\) &  &  \\
                \hline
                \(\downarrow\)W & \(\times\) & \(\times\) &  &  \\
                \hline
            \end{tabular}
            \begin{tabular}{l}
                \(\times\): should be UB \\
                \(\approx\): should be UB earlier \\
            \end{tabular}
        \end{center}

        Concept adapted from Stacked Borrows: protectors.
        \begin{itemize}
            \item references get a protector on function entry
            \item protector lasts until the end of the call
            \item protectors strengthen the guarantees
        \end{itemize}
    \end{onlyenv}
    \includegraphics<2>{blank.prot.pdf}
    \includegraphics<3>{steps.prot.cp.pdf}
    \includegraphics<4>{steps.prot.nodis.pdf}
    \includegraphics<5>{steps.prot.noalias.pdf}
\end{frame}

\begin{frame}[fragile]
    \frametitle{Protectors lock permissions}
    \begin{alertblock}{LLVM \texttt{noalias} (in TB terms)}
        No foreign access during the same function call as a child write.
    \end{alertblock}

    \begin{block}{With protectors: fixed}
        \begin{lstlisting}[language=rust, escapechar=@]
fn write(x: &mut u64) { // with protector
    *x = 42; // activation
    opaque(/* foreign read for x: noalias violation */);
}
        \end{lstlisting}
    \end{block}
    \includegraphics[width=0.7\textwidth]{path.prot.mut+cw+fr.pdf}
\end{frame}

%\begin{frame}[fragile, t]
%    \frametitle{Why not just...}
%    \begin{block}{...insert an implicit read/write on function exit ?}
%        \begin{lstlisting}[language=rust, escapechar=@]
%fn write(x: &mut u64) { // with protector
%    *x = 42; // activation
%    opaque(/* foreign read for x: noalias violation */);
%    // implicit write through x ?
%}
%        \end{lstlisting}
%    \end{block}
%    \begin{onlyenv}<1>
%        \includegraphics[width=0.7\textwidth]{path.base.mut+cw+fr.pdf}
%    \end{onlyenv}
%    \begin{onlyenv}<2>
%        \begin{itemize}
%            \item what if \texttt{opaque} doesn't terminate ?
%            \item should we insert a read or a write ?
%        \end{itemize}
%    \end{onlyenv}
%\end{frame}


\subsection*{}

\begin{frame}
    \frametitle{Summary}
    \begin{itemize}
        \item \texttt{Reserved}, \texttt{Active}, \texttt{Frozen}, \texttt{Disabled}
            represent different possible states of pointers.\\
        \item Interactions with child and foreign accesses enforce uniqueness/immutability guarantees.
        \item Protectors are added on function entry to strengthen these guarantees up to the
            requirements of \texttt{noalias}.\\
    \end{itemize}
\end{frame}
