\section{Deriving rules}

% first Active, Frozen, Disabled
% then Reserved
% then protectors
% then interiormut

\begin{frame}
    \frametitle{Life of a permission}

    \only<1->{\([p_0, p_0, p_0, p_0, p_0, p_0, ...]\)}
    \only<2->{+\text{foreign write [1..=3]}}~\\
        \only<2->{\([p_0, {\color{blue}p_1}, {\color{blue}p_1}, {\color{blue}p_1}, p_0, p_0, ...]\)}
    \only<3->{+\text{child read [0..=1]}}~\\
        \only<3->{\([{\color{blue}p_2}, {\color{blue}p_3}, p_1, p_1, p_0, p_0, ...]\)}
    \only<4->{+\text{child read [0..=2]}}~\\
        \only<4->{\([{\color{blue}p_2}, {\color{blue}p_3}, {\color{blue}p_3}, p_1, p_0, p_0, ...]\)}
    \only<5->{+\text{child write [3..=4]}}~\\
        \only<5->{\([p_2, p_3, p_3, {\color{blue}p_4}, {\color{blue}p_5}, p_0, ...]\)}
    \only<6->{+\text{foreign write [2..=4]}}~\\
        \only<6->{\([p_2, p_3, {\color{blue}p_6}, {\color{blue}p_7}, {\color{blue}p_8}, p_0, ...]\)}
        \only<6->{\(\text{UB at location [3]: } p_4 \to p_7\)}~\\
\end{frame}

\begin{frame}[t]
    \frametitle{Active, Frozen, Disabled}
    At the very least we need permissions to represent
    \begin{itemize}
        \item unique mutable references: \texttt{Active},
        \item shared immutable references: \texttt{Frozen},
        \item no longer usable: \texttt{Disabled}.
    \end{itemize}

    \begin{onlyenv}<1>
        \begin{minipage}{0.3\textwidth}
            \begin{block}{\texttt{Active}}
                \begin{itemize}
                    \item child read OK
                    \item child write OK
                \end{itemize}
            \end{block}
        \end{minipage}
        ~
        \begin{minipage}{0.3\textwidth}
            \begin{block}{\texttt{Frozen}}
                \begin{itemize}
                    \item child read OK
                    \item child write UB
                \end{itemize}
            \end{block}
        \end{minipage}
        ~
        \begin{minipage}{0.3\textwidth}
            \begin{block}{\texttt{Disabled}}
                \begin{itemize}
                    \item child read UB
                    \item child write UB
                \end{itemize}
            \end{block}
        \end{minipage}
    \end{onlyenv}

    \begin{onlyenv}<2>
        \begin{block}{Foreign read}
            \begin{itemize}
                \item \texttt{Disabled}, \texttt{Frozen}: nothing to do
                \item \texttt{Active}: no longer unique\\
                      most permissive solution: make it become \texttt{Frozen}
            \end{itemize}
        \end{block}

        \begin{block}{Foreign write}
            \begin{itemize}
                \item \texttt{Disabled}: nothing to do
                \item \texttt{Active}: no longer unique, must become \texttt{Disabled}
                \item \texttt{Frozen}: no longer immutable, must become \texttt{Disabled}
            \end{itemize}
        \end{block}
    \end{onlyenv}
\end{frame}

\begin{frame}
    \frametitle{Parallel to the borrow checker}
    Similarities
    \hspace{-5em}
    \begin{description}
        \item[\cmark] \texttt{\&mut} readable and writeable
        \item[\cmark] \texttt{\&} and all their children are only readable
        \item[\cmark] no one can read/write to the location of our \texttt{\&mut}
        \item[\cmark] no one can write to the location of our \texttt{\&}
    \end{description}

    Differences
    \hspace{-5em}
    \begin{description}
        \item[\xmark] \texttt{\&mut} demoted to \texttt{\&}
        \item[\xmark] several \texttt{\&mut} can coexist if never written to
    \end{description}
\end{frame}

\begin{frame}[t]
    \frametitle{Raw pointers}
    \begin{block}{Mutability \textit{and} aliasing ?}
        Make them all the ``same'' pointer: same tag.
        \begin{itemize}
            \item arbitrary usage order between pointers with the same tag
            \item all copies lose permissions at the same time as the original
        \end{itemize}
    \end{block}

    \begin{onlyenv}<2>
        \begin{block}{\texttt{UnsafeCell}}
            Similar approach: \texttt{\&UnsafeCell<T>} is treated as a raw pointer.
        \end{block}

        \begin{exampleblock}{WIP: finer grained version ?}
            Thoughts of introducing new permission to handle per-location interior mutability
        \end{exampleblock}
    \end{onlyenv}
\end{frame}

\subsection{Two-phase borrows}

\begin{frame}[fragile, t]
    \frametitle{Not all mutable references can be \texttt{Active}}
    \begin{block}{}
        \begin{lstlisting}[language=rust, escapechar=\@]
@              @fn main() {
@\visible<2>{>}@    let mut v = vec![0usize, 1, 2, 3];
@\visible<3,5>{>}@    v.push(
@\visible<4,5>{>}@        v.len()
@\visible<5>{>}@    );
@              @}
        \end{lstlisting}
    \end{block}
    \begin{block}{}
        \begin{tikzpicture}[
            every node/.append style = {anchor = west},
            grow via three points={one child at (0.3,-0.5) and two children at (0.3,-0.5) and (0.3,-1.0)},
            edge from parent path={(\tikzparentnode\tikzparentanchor) |- (\tikzchildnode\tikzchildanchor)}
            ]
            \node (nv) at (0,0) {\texttt{v:}}
                    child {node (nvpush) {\texttt{v\textsubscript{push}:}}}
                    child {node (nvlen) {\texttt{v\textsubscript{len}:}}};

            \node<2->[right of=nv] {\texttt{Active}};
            \node<3>[right of=nvpush] {~~\texttt{Active}};
            \node<4->[right of=nvpush] {~~\texttt{Frozen}};
            \node<4->[right of=nvlen] {~~\texttt{Frozen}};

            \node (vert) at (5,0 |- nv) {};
            \node<3,4> at (vert |- nv) {\(\gets\) reborrow};
            \node<4> at (vert |- nvlen) {\(\gets\) read};
            \node<5> at (vert |- nvpush) {\(\gets\) write};
        \end{tikzpicture}
    \end{block}
    \begin{onlyenv}<5>
        \begin{block}{}
            \texttt{push} attempts a write through \texttt{Frozen}.
        \end{block}
    \end{onlyenv}
\end{frame}

\begin{frame}
    \frametitle{Behavior of \texttt{Reserved}}
    \begin{itemize}
        \item a mutable reference not yet written to
        \item allows child reads, becomes \texttt{Active} on the first child write
        \item allows foreign reads, becomes \texttt{Disabled} on the first foreign write
    \end{itemize}
    \(\to\) behaves as a \texttt{Frozen} until the first child write\\
    \(\to\) can coexist with each other and with \texttt{Frozen}\\

    ~\\
    In Tree Borrows we choose to make all mutable reborrows initially \texttt{Reserved},
    not just two-phase ones.
\end{frame}

\begin{frame}[fragile, t]
    \frametitle{\texttt{Reserved} in action}
    \begin{block}{}
        \begin{lstlisting}[language=rust, escapechar=\@]
@              @fn main() {
@\visible<2>{>}@    let mut v = vec![0usize, 1, 2, 3];
@\visible<3,5>{>}@    v.push(
@\visible<4,5>{>}@        v.len()
@\visible<5>{>}@    );
@              @}
        \end{lstlisting}
    \end{block}
    \begin{block}{}
        \begin{tikzpicture}[
            every node/.append style = {anchor = west},
            grow via three points={one child at (0.3,-0.5) and two children at (0.3,-0.5) and (0.3,-1.0)},
            edge from parent path={(\tikzparentnode\tikzparentanchor) |- (\tikzchildnode\tikzchildanchor)}
            ]
            \node (nv) at (0,0) {\texttt{v:}}
                    child {node (nvpush) {\texttt{v\textsubscript{push}:}}}
                    child {node (nvlen) {\texttt{v\textsubscript{len}:}}};

            \node<2->[right of=nv] {\texttt{Active}};
            \node<3,4>[right of=nvpush] {~~~~~\texttt{Reserved}};
            \node<5>[right of=nvpush] {~~\texttt{Active}};
            \node<4>[right of=nvlen] {~~\texttt{Frozen}};
            \node<5>[right of=nvlen] {~~~~\texttt{Disabled}};

            \node (vert) at (5,0 |- nv) {};
            \node<3,4> at (vert |- nv) {\(\gets\) reborrow};
            \node<4> at (vert |- nvlen) {\(\gets\) read};
            \node<5> at (vert |- nvpush) {\(\gets\) write};
        \end{tikzpicture}
    \end{block}
    \begin{onlyenv}<5>
        \begin{block}{}
            Write through \texttt{Reserved} activates it into \texttt{Active}.
        \end{block}
    \end{onlyenv}
\end{frame}

\subsection{Justifying \texttt{noalias}}

\begin{frame}
    \frametitle{Reads on reborrow}
    We must \textit{at least} do a read access on a reborrow: \texttt{dereferenceable}.\\

    \begin{block}{Note}
        On mutable reborrows (creation of \texttt{Reserved}) we only perform a read,
        which is strictly more permissive than Stacked Borrows.
    \end{block}

    \begin{block}{Note}
        One option to make Tree Borrows more restrictive (and allow more optimizations)
        is to make non-two-phase mutable borrows \texttt{Active} (and write on reborrow).
    \end{block}
\end{frame}

\begin{frame}[fragile]
    \frametitle{The need for protectors}
    \begin{onlyenv}<1-3>
        \begin{block}{Function entry reborrows need something more}
            \begin{lstlisting}[language=rust, escapechar=@]
@ @fn main() {
@ @    let mut data = 0;
@ @    unsafe { copy(
@\visible<2>{>}@        &mut *addr_of_mut!(data),
@\visible<2>{>}@        &*addr_of!(data),
@ @    ); }
@ @}

@ @fn copy(x: &mut u64, y: &u64) {
@\visible<3>{>}@    let val = *y; // x: Reserved
@\visible<3>{>}@    *x = val; // x: Active
@ @}

            \end{lstlisting}
        \end{block}
    \end{onlyenv}

    \begin{onlyenv}<4>
        Concept adapted from Stacked Borrows: protectors.
        \begin{itemize}
            \item references get a protector on function entry
            \item protector lasts until the end of the call
        \end{itemize}
        ~\\
        While protected
        \begin{itemize}
            \item behavior changes
                \begin{itemize}
                    \item \(\texttt{Reserved} \to \texttt{Frozen}\) on foreign write
                \end{itemize}
            \item some transitions trigger UB
                \begin{itemize}
                    \item \(\texttt{\_} \to \texttt{Disabled}\)
                    \item \(\texttt{Active} \to \texttt{Frozen}\)
                \end{itemize}
        \end{itemize}

        \begin{block}{}
            Protectors provide enough guarantees to justify \texttt{noalias} and
            \texttt{dereferenceable} on references in function arguments.
        \end{block}
    \end{onlyenv}
\end{frame}

\begin{frame}
    \frametitle{Summary}
    \begin{itemize}
        \item \texttt{Reserved}, \texttt{Active}, \texttt{Frozen}, \texttt{Disabled}
            represent different possible states of pointers
            \begin{itemize}
                \item all but \texttt{Disabled} allow child reads
                \item \texttt{Active} and \texttt{Reserved} enable child writes
                \item everything becomes \texttt{Disabled} on a foreign write
                \item \texttt{Active} becomes \texttt{Frozen} on a foreign read
            \end{itemize}
        \item protectors are added on function entry and strengthen the guarantees
            of \texttt{Reserved}, \texttt{Active} and \texttt{Frozen} to guarantee \texttt{noalias}
            and \texttt{dereferenceable}
    \end{itemize}
\end{frame}
