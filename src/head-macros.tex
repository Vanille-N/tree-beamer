% Ce fichier contient toutes les macros que vous pouvez avoir envie de définir
% si vous les utilisez plusieurs fois dans le document.

\PassOptionsToPackage{svgnames}{color}

% Un environnement pour bien présenter le code informatique
\newenvironment{code}{%
\begin{mdframed}[linecolor=green,innerrightmargin=30pt,innerleftmargin=30pt,
backgroundcolor=black!5,
skipabove=10pt,skipbelow=10pt,roundcorner=5pt,
splitbottomskip=6pt,splittopskip=12pt]
}{%
\end{mdframed}
}

\newcommand{\bijective}{%
  \hookrightarrow\mathrel{\mspace{-15mu}}\rightarrow
}
\newcommand{\surjective}{\twoheadrightarrow}
\newcommand{\injective}{\hookrightarrow}
\newcommand{\implication}{\Longrightarrow}
\newcommand{\impl}{\Rightarrow}
\newcommand{\reciprocal}{\Longleftarrow}
\newcommand{\equivalent}{\Longleftrightarrow}
\newcommand{\NN}{\ensuremath{\mathbb{N}}}
\newcommand{\RR}{\ensuremath{\mathbb{R}}}
\newcommand{\QQ}{\ensuremath{\mathbb{Q}}}
\newcommand{\ZZ}{\ensuremath{\mathbb{Z}}}
\newcommand{\CC}{\ensuremath{\mathbb{C}}}
\newcommand{\EE}{\ensuremath{\mathbb{E}}}
\newcommand{\PP}{\ensuremath{\mathbb{P}}}
\renewcommand{\epsilon}{\varepsilon}
\renewcommand{\phi}{\varphi}
\renewcommand{\leq}{\leqslant}
\renewcommand{\geq}{\geqslant}

\newcommand{\bbrack}[1]{\left\llbracket#1\right\rrbracket}

\newcommand{\inv}[1]{#1^{-1}}

\newcommand{\tsf}[1]{\textsf{#1}}
\newcommand{\ttt}[1]{\texttt{#1}}
\newcommand{\tbf}[1]{\textbf{#1}}
\newcommand{\tsc}[1]{\textsc{#1}}
\newcommand{\tq}{\ |\ }
\newcommand{\abs}[1]{\ensuremath{\left|#1\right|}}
\newcommand{\set}[1]{\ensuremath{\left\{#1\right\}}}
\newcommand{\paren}[1]{\ensuremath{\left(#1\right)}}

\newcommand\semantics{%
    \mathrel{\rightsquigarrow}}


\newlength{\mirageWidth}\newlength{\mirageHeight}
\newcommand{\mirage}[1]{% The opposite of a \phantom: it is visible but it takes no space
    #1%
    \settowidth{\mirageWidth}{#1}%
    \settoheight{\mirageHeight}{#1}%
    \hspace{-\mirageWidth}%
    \vspace{-\mirageHeight}%
}

\DeclareMathOperator{\poly}{poly}
\DeclareMathOperator{\expn}{exp}
\DeclareMathOperator{\minvolume}{\exists\textup{\tsf{MVol}}}
\DeclareMathOperator{\avgvolume}{\textup{\tsf{EMVol}}}
\DeclareMathOperator{\maxvolume}{\textup{\tsf{DMVol}}}
\DeclareMathOperator{\radius}{\textup{\tsf{MRad}}}
\newcommand{\neigh}{\mathcal{N}}

\newenvironment{stacked}{%
    \begin{tikzpicture}%
        \node (ref) {};%
}{%
    \end{tikzpicture}%
}

\newcommand{\stackitem}[2]{%
    \node at (ref) {{%
        \visible<#1>{%
            \begin{minipage}{\textwidth}
                #2%
            \end{minipage}
        }%
    }};%
}

\newcommand{\runnable}[2]{%
    \begin{block}{Demo}%
        \href{run:run/#1.sh}{\texttt{#2}}%
    \end{block}%
}

\newcommand{\cmark}{\color{darkgreen}{\ding{51}}} % check mark
\newcommand{\xmark}{\color{red}{\ding{55}}} % cross mark
