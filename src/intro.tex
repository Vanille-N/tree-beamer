\begin{frame}
    \titlepage
\end{frame}

\section{Introduction}

% VERB: impossible to reorder reads. does the compiler reorder reads ? Is compiler + SB unsound ?
% VERB: explain why the tree is structural; provide intuition why it's also the good structure
% FIXME: rework all summary slides

\begin{frame}[fragile,t]
    \frametitle{A motivating example}
    See code example: \texttt{demo}
\end{frame}

\begin{frame}
    \frametitle{Is my optimization unsound ?\\No, it's the client that is UB.}
    UB: ``Undefined Behavior''
    \begin{itemize}
        \item the semantics of a program that contains UB are undefined: any behavior can occur
        \item equivalently: the compiler can assume that UB does not occur
    \end{itemize}~\\~\\

    The compiler is allowed to ``miscompile'' programs that contain UB.~\\~\\

    In this case: pointer aliasing UB.\\
    (Other kinds: uninitialized memory, data races, dangling pointers, invalid values, ...)
\end{frame}

\begin{frame}
    \frametitle{The role of pointer aliasing UB}
    (Tree|Stacked) Borrows
    \begin{itemize}
        \item \textbf{define} an operational semantics with UB of reborrows and memory accesses
        \item \textbf{detect} violations of the aliasing discipline it dictates
    \end{itemize}
    in order to
    \begin{itemize}
        \item \textbf{enable compiler optimizations} by ruling out aliasing patterns
            \begin{itemize}
                \item remove redundant loads and stores
                \item permute noninterfering operations
            \end{itemize}
        \item \textbf{hint at bugs} in \texttt{unsafe} code
        \item \textbf{justify LLVM attributes} on pointers that \texttt{rustc} emits
            \begin{itemize}
                \item \texttt{noalias}: added by rustc on references, means that the data
                    is not being mutated through several different pointers
                \item \texttt{dereferenceable}: added by rustc on references, means that
                    the pointer is not null or dangling or otherwise invalid to dereference
            \end{itemize}
    \end{itemize}~\\
\end{frame}

\begin{frame}[fragile]
    \frametitle{Motivating example: with Miri}
    Miri (\href{https://github.com/rust-lang/miri}{\texttt{github:rust-lang/miri}}) is
    \begin{itemize}
        \item a Rust interpreter
        \item that detects UB
    \end{itemize}~\\

    Back to \texttt{demo}: run with Miri
\end{frame}

\begin{frame}
    \frametitle{Basics of Stacked Borrows (SB)}
    \begin{block}{Starting observation}
        Proper usage of mutable references follows a stack discipline.
    \end{block}
    \begin{block}{Key ideas}
        \begin{itemize}
            \item per-location tracking of pointers
            \item use a stack to store pointer identifiers
            \item on each reborrow a new identifier is pushed to the stack
            \item a pointer can be used if its identifier is in the stack
            \item using of a pointer pops everything above it (more recent)
        \end{itemize}
    \end{block}
\end{frame}

\begin{frame}[fragile]
    \frametitle{An example SB execution}
    \begin{block}{}
        \begin{lstlisting}[language=rust]
let x = &mut 0; // [x]
let y = &mut *x; // [x, y] (reborrow of x into y)
let z = &mut *x; // [x, z] (usage of x pops y)
                   // (reborrow of x into z)
*y = 42; // y is not in the stack, UB !
*z = 57;
        \end{lstlisting}
    \end{block}
\end{frame}

\begin{frame}
    \frametitle{SB too strict ?}
    Many optimizations are now possible again, but...~\\

    UB in \texttt{tokio}, \texttt{pyo3}, \texttt{rkyv}, \texttt{eyre},
    \texttt{ndarray}, \texttt{arrayvec}, \texttt{slotmap}, \texttt{nalgebra},
    \texttt{json}, ...~\\
    Enforcing SB would break backwards compatibility, so right now the compiler
    cannot do any SB-related optimizations.

    \begin{block}{Essential tradeoff}
        More UB is...
        \begin{itemize}
            \item more optimizations (stronger assumptions)
            \item less safety (especially if rules are vague)
        \end{itemize}
    \end{block}
    \begin{block}{}
        UB is the responsibility of the user (the compiler doesn't care about it),
        so \textbf{too much UB makes users unhappy}.
    \end{block}
\end{frame}

\begin{frame}[fragile, t]
    \frametitle{Information loss in the stack}
    \begin{block}{}
        \[[a, b, c, d]\]
    \end{block}

    \begin{minipage}{0.3\textwidth}
        \begin{block}{}
            \begin{lstlisting}[language=rust]
let a = &...;
let b = &*a;
let c = &*b;
let d = &*a;
            \end{lstlisting}

            \begin{tikzpicture}[
                every node/.append style = {anchor = west},
                grow via three points={one child at (0.5,-0.8) and two children at (0.5,-0.8) and (0.5,-1.6)},
                edge from parent path={(\tikzparentnode\tikzparentanchor) |- (\tikzchildnode\tikzchildanchor)}]
                \node[draw] at (0,0) {\(a\)}
                    child {node[draw] {\(b\)}
                        child {node[draw] {\(c\)}}}
                    child [missing] {}
                    child[draw] {node[draw] {\(d\)}};
            \end{tikzpicture}
        \end{block}
    \end{minipage}
    ~
    \begin{minipage}{0.3\textwidth}
        \begin{block}{}
            \begin{lstlisting}[language=rust]
let a = &...;
let b = &*a;
let c = &*a;
let d = &*a;
            \end{lstlisting}

            \begin{tikzpicture}[
                every node/.append style = {anchor = west},
                grow via three points={one child at (0.5,-0.8) and two children at (0.5,-0.8) and (0.5,-1.6)},
                edge from parent path={(\tikzparentnode\tikzparentanchor) |- (\tikzchildnode\tikzchildanchor)}]
                \node[draw] at (2.7,0) {\(a\)}
                    child {node[draw] {\(b\)}}
                    child {node[draw] {\(c\)}}
                    child {node[draw] {\(d\)}};
            \end{tikzpicture}
        \end{block}
    \end{minipage}
    ~
    \begin{minipage}{0.3\textwidth}
        \begin{block}{}
            \begin{lstlisting}[language=rust]
let a = &...;
let b = &*a;
let c = &*b;
let d = &*c;
            \end{lstlisting}

            \begin{tikzpicture}[
                every node/.append style = {anchor = west},
                grow via three points={one child at (0.5,-0.8) and two children at (0.5,-0.8) and (0.5,-1.6)},
                edge from parent path={(\tikzparentnode\tikzparentanchor) |- (\tikzchildnode\tikzchildanchor)}]
                \node[draw] at (2.7,0) {\(a\)}
                    child {node[draw] {\(b\)}
                        child {node[draw] {\(c\)}
                            child {node[draw] {\(d\)}}}};
            \end{tikzpicture}
        \end{block}
    \end{minipage}
\end{frame}

\begin{frame}[t]
    \frametitle{\cancel{Stacked} Tree Borrows (TB)}
    \begin{onlyenv}<1>
        \begin{block}{Starting observation}
            Proper usage of mutable references follows a stack discipline.
        \end{block}
        \begin{block}{Key ideas}
            \begin{itemize}
                \item per-location tracking of pointers
                \item use a stack to store pointer identifiers
                \item on each reborrow a new identifier is pushed to the top of the stack
                \item a pointer can be used if its identifier is in the stack
                \item using of a pointer pops everything above it (more recent)
            \end{itemize}
        \end{block}
    \end{onlyenv}

    \begin{onlyenv}<2>
        \begin{block}{Starting observation}
            Proper usage of {\color{red}all pointers} follows a {\color{red}tree} discipline.
        \end{block}
        \begin{block}{Key ideas}
            \begin{itemize}
                \item per-location tracking of pointers
                \item use a {\color{red}tree} to store pointer identifiers
                \item on each reborrow a new identifier is {\color{red}added as a leaf of the tree}
                \item {\color{red}each pointer has permissions}
                \item a pointer can be used {\color{red}if its permission allows it (to be defined)}
                \item using a pointer {\color{red}makes incompatible (to be defined) pointers lose permissions}
            \end{itemize}
        \end{block}
    \end{onlyenv}
\end{frame}

