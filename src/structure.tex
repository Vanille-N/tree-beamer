\section{Tree Structure}

% VERB: Are you also keeping track of the well-formedness of the tree somehow? 
% VERB: Is it an assumption or do you throw an error if you somehow detect a cycle?

\begin{frame}[fragile]
    \frametitle{When are pointers different ?}
    LLVM and Rust specifications talk about ``other references/pointers'', but what does that mean ?

    Suggests that two pointers to the same data are ``different''.

    A pointer in our semantics is:
    \begin{lstlisting}[language=rust]
struct Pointer {
    address: usize,
    size: usize,
    tag: usize, // <- added specifically for TB/SB
}
    \end{lstlisting}~\\

    Even if two pointers point to the same data, they are not equal for TB
    if they have different tags.
    On the other hand, TB does not care about the size or the type.
\end{frame}

\begin{frame}[fragile]
    \frametitle{A Tree of pointers}
    \begin{minipage}{0.45\textwidth}
        \begin{block}{}
            \begin{tikzpicture}[
                every node/.append style = {anchor = west},
                grow via three points={one child at (0.5,-0.8) and two children at (0.5,-0.8) and (0.5,-1.6)},
                edge from parent path={(\tikzparentnode\tikzparentanchor) |- (\tikzchildnode\tikzchildanchor)}]
            \node at (0,0) {\texttt{x}}
                child {node {\texttt{y1}}
                    child {node {\texttt{y2}}}}
                child[missing] {}
                child {node {\texttt{z1}, \texttt{z2}}}
                child {node {\texttt{w1}}
                    child {node {\texttt{w2}}
                        child {node {\texttt{w3}}}}};
            \end{tikzpicture}
        \end{block}
    \end{minipage}
    ~\ ~\
    \begin{minipage}{0.45\textwidth}
        \begin{block}{}
            \begin{lstlisting}[language=rust]
let x = &mut 0u64;

let y1 = &mut *x;
let y2 = &*y1;

let z1 = &*x;
let z2 = z1 as *mut u64;

foo(&mut *x);
fn foo(w2: &mut u64) {
    let w3 = &*w2;
}

            \end{lstlisting}
        \end{block}
    \end{minipage}
\end{frame}

\begin{frame}
    \frametitle{What's in the tree ?}
    Each pointer is given a tag (opaque, impossible to forge, primitive operation to create a fresh tag)
    ~\\
    What (Tree|Stacked) Borrows track:
    \begin{itemize}
        \item \textbf{permission} of each tag for each location;
        \item some \textbf{structure} between those tags;
        \item all accesses are done through a tag:
            \begin{itemize}
                \item read/write accesses \textbf{require permissions} on the tag used,\\
                    UB occurs if the permissions are insufficient;
                \item read/write accesses \textbf{alter permissions} of other tags,\\
                    UB occurs if the modification is forbidden;
            \end{itemize}
    \end{itemize}
\end{frame}

\begin{frame}
    \frametitle{One pointer, \(2 \times 2\) kinds of accesses}
    \begin{block}{}
        \centering
        \begin{tikzpicture}[
            every node/.append style = {anchor = west},
            grow via three points={one child at (0.3,-0.5) and two children at (0.3,-0.5) and (0.3,-1.0)},
            edge from parent path={(\tikzparentnode\tikzparentanchor) |- (\tikzchildnode\tikzchildanchor)}]
            \node (root) at (0,0) {\texttt{?}}
                child {node (nfather) {\texttt{?}}
                    child {node (nxfirst) {\color{red}\texttt{x}}
                        child {node {\texttt{?}}}
                        child {node {\texttt{?}}
                            child {node (nxlast) {\texttt{?}}}}}
                    child[missing] {}
                    child[missing] {}
                    child[missing] {}
                    child {node (nnextfirst) {\texttt{?}}}
                    child {node {\texttt{?}}
                        child {node (nnextlast) {\texttt{?}}}}}
                ;
            \draw (2.5,0 |- root) node (vmark) {};
            \draw [decorate, decoration = {brace}] (vmark |- nxfirst) -- (vmark |- nxlast)
                node[midway] {~\textit{child} accesses for \color{red}\texttt{x}};
            \draw [decorate, decoration = {brace}] (vmark |- root) -- (vmark |- nfather)
                node[midway] {~\textit{foreign} accesses for \color{red}\texttt{x}};
            \draw [decorate, decoration = {brace}] (vmark |- nnextfirst) -- (vmark |- nnextlast)
                node[midway] {~\textit{foreign} accesses for \color{red}\texttt{x}};
        \end{tikzpicture}
        ~\\
        each \textit{read} or \textit{write}.
    \end{block}
    \begin{block}{``pointers based on...''}
        LLVM specification: ``pointer B is based on pointer A'' \(\simeq\) TB: ``B is a child of A''.
        (\sout{SB: B was created before A})
    \end{block}
\end{frame}

\begin{frame}[fragile]
    \frametitle{Kinds of accesses: examples}
    \begin{block}{}
        \begin{lstlisting}[language=rust]
let x = &mut ...;
let y = &mut *x;

*x = 1; // Write access; foreign for y; child for x.
let _ = *y; // Read access; child for y; child for x.
        \end{lstlisting}
    \end{block}
    \begin{block}{}
        \centering
        \begin{tikzpicture}[
            every node/.append style = {anchor = west},
            grow via three points={one child at (0.3,-0.5) and two children at (0.3,-0.5) and (0.3,-1.0)},
            edge from parent path={(\tikzparentnode\tikzparentanchor) |- (\tikzchildnode\tikzchildanchor)}]
            \node {...}
                child {node {\texttt{x}}
                    child {node {\texttt{y}}}};
        \end{tikzpicture}
    \end{block}
\end{frame}

\begin{frame}
    \frametitle{Summary}
    % FIXME: rework
    \begin{itemize}
        \item pointers are identified by a tag, and tags are stored in a tree structure;
        \item a reborrow of \(t\) either...
            \begin{itemize}
                \item creates a new tag \(t'\), \(t'\) is recorded as a child of \(t\), or
                \item keeps the same tag \(t\).
            \end{itemize}
        \item each tag has a permission on each location:
            \begin{itemize}
                \item accesses through child tags are \textit{child accesses},
                \item accesses through non-child tags are \textit{foreign accesses},
                \item all accesses can cause a modification of the permission, and some modifications are indicators of UB.
            \end{itemize}
    \end{itemize}
\end{frame}
