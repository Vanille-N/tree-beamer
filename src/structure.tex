\section{Why a tree ?}

\begin{frame}
    \frametitle{What's in the tree ?}
    Pointers, mostly.

    \begin{itemize}
        \item each pointer is given a tag;
        \item opaque, impossible to forge;
        \item anyone can create a fresh tag: retagging;
    \end{itemize}
    ~\\
    What (Tree|Stacked) Borrows track:
    \begin{itemize}
        \item \textbf{permission} of each tag for each location;
        \item some \textbf{structure} between those tags;
        \item all accesses are done through a tag:
            \begin{itemize}
                \item read/write accesses \textbf{require permissions} on the tag used,\\
                    UB occurs if the permissions are insufficient;
                \item read/write accesses \textbf{alter permissions} of other tags,\\
                    UB occurs if the modification is forbidden;
            \end{itemize}
    \end{itemize}
\end{frame}

\begin{frame}
    \frametitle{Indistinguishable structures: trees vs stacks}
    \begin{minipage}{0.45\textwidth}
        \begin{block}{}
            \[[a, b, c, d] \ne [a, c, d, b]\]
            \centering
            \begin{tikzpicture}[
                every node/.append style = {anchor = west},
                grow via three points={one child at (0.5,-0.8) and two children at (0.5,-0.8) and (0.5,-1.6)},
                edge from parent path={(\tikzparentnode\tikzparentanchor) |- (\tikzchildnode\tikzchildanchor)}]
            \node[draw] at (0,0) {\(a\)}
                child {node[draw] {\(b\)}}
                child {node[draw] {\(c\)}}
                child {node[draw] {\(d\)}};
            \node at (1.5,-0.5) {\(\simeq\)};
            \node[draw] at (2.5,0) {\(a\)}
                child {node[draw] {\(c\)}}
                child {node[draw] {\(d\)}}
                child {node[draw] {\(b\)}};
            \end{tikzpicture}
        \end{block}
    \end{minipage}
    ~\ ~\
    \begin{minipage}{0.45\textwidth}
        \begin{block}{}
            \[[a, b, c, d] \simeq [a, b, c, d]\]
            \centering
            \begin{tikzpicture}[
                every node/.append style = {anchor = west},
                grow via three points={one child at (0.5,-0.8) and two children at (0.5,-0.8) and (0.5,-1.6)},
                edge from parent path={(\tikzparentnode\tikzparentanchor) |- (\tikzchildnode\tikzchildanchor)}]
                \node[draw] at (0,0) {\(a\)}
                    child {node[draw] {\(b\)}}
                    child {node[draw] {\(c\)}
                        child[draw] {node[draw] {\(d\)}}};
                \node at (2,-0.5) {\(\ne\)};
                \node[draw] at (2.7,0) {\(a\)}
                    child {node[draw] {\(b\)}
                        child {node[draw] {\(c\)}}}
                    child [missing] {}
                    child {node[draw] {\(d\)}};
            \end{tikzpicture}
        \end{block}
    \end{minipage}
\end{frame}

\begin{frame}[fragile]
    \frametitle{All strict child accesses are the same}
    \begin{minipage}{0.45\textwidth}
        \begin{block}{}
            {\tiny
            \begin{lstlisting}[language=rust, basicstyle=\ttfamily\scriptsize]
fn reborrow(x: &u64) -> &u64 {
    &*x
}
            \end{lstlisting}
            }
            {\small
            \begin{tikzpicture}[
                every node/.append style = {anchor = west},
                grow via three points={one child at (0.3,-0.5) and two children at (0.3,-0.5) and (0.3,-1.0)},
                edge from parent path={(\tikzparentnode\tikzparentanchor) |- (\tikzchildnode\tikzchildanchor)}]
            \node (nxcaller) at (0,0) {\texttt{x}}
                child {node (nxcallee) {\texttt{x}}
                    child {node {\texttt{\&*x}}
                        child {node {\texttt{ret}}
                            child {node {} edge from parent [draw=none]}}}};
            \node[right of=nxcaller] {\tiny(caller)};
            \node[right of=nxcallee] {\tiny(callee)};
            \end{tikzpicture}
            }
        \end{block}
    \end{minipage}
    ~\ ~\
    \begin{minipage}{0.45\textwidth}
        \begin{block}{}
            {\tiny
            \begin{lstlisting}[language=rust, basicstyle=\ttfamily\scriptsize]
fn reborrow(x: &u64) -> &u64 {
    &*&*x
}
            \end{lstlisting}
            }
            {\small
            \begin{tikzpicture}[
                every node/.append style = {anchor = west},
                grow via three points={one child at (0.3,-0.5) and two children at (0.3,-0.5) and (0.3,-1.0)},
                edge from parent path={(\tikzparentnode\tikzparentanchor) |- (\tikzchildnode\tikzchildanchor)}]
            \node (nxcaller) at (0,0) {\texttt{x}}
                child {node (nxcallee) {\texttt{x}}
                    child {node {\texttt{\&*x}}
                        child {node {\texttt{\&*\&*x}}
                            child {node {\texttt{ret}}}}}};
            \node[right of=nxcaller] {\tiny(caller)};
            \node[right of=nxcallee] {\tiny(callee)};
            \end{tikzpicture}
            }
        \end{block}
    \end{minipage}

    \begin{block}{}
        {\small
            \begin{tikzpicture}[
                every node/.append style = {anchor = west},
                grow via three points={one child at (0.3,-0.5) and two children at (0.3,-0.5) and (0.3,-1.0)},
                edge from parent path={(\tikzparentnode\tikzparentanchor) |- (\tikzchildnode\tikzchildanchor)}]
            \node (nx) at (0,0) {\texttt{x}}
                    child {node {\texttt{?}}
                        child {node {\texttt{ret}}}};
            \node[right of=nx] {\tiny(caller)};
            \end{tikzpicture}
        }
    \end{block}
\end{frame}

\begin{frame}[fragile]
    \frametitle{All non-child accesses are the same}
        \begin{minipage}{0.28\textwidth}
        \begin{block}{}
            {\tiny
            \begin{lstlisting}[language=rust, basicstyle=\ttfamily\scriptsize]
let y = &*&*x;
x.do_something();
            \end{lstlisting}
            }
            {\small
                \begin{tikzpicture}[
                    every node/.append style = {anchor = west},
                    grow via three points={one child at (0.3,-0.5) and two children at (0.3,-0.5) and (0.3,-1.0)},
                    edge from parent path={(\tikzparentnode\tikzparentanchor) |- (\tikzchildnode\tikzchildanchor)}]
                \node (nx) at (0,0) {\texttt{x}}
                    child {node {\texttt{\&*x}}
                        child {node {\texttt{y}}}};
                \node[right of=nx] {\(\gets \text{access}\)};
                \end{tikzpicture}
            }
        \end{block}
    \end{minipage}
    ~\ ~\
    \begin{minipage}{0.28\textwidth}
        \begin{block}{}
            {\tiny
            \begin{lstlisting}[language=rust, basicstyle=\ttfamily\scriptsize]
let y = &*x;
x.do_something();
            \end{lstlisting}
            }
            {\small
                \begin{tikzpicture}[
                    every node/.append style = {anchor = west},
                    grow via three points={one child at (0.3,-0.5) and two children at (0.3,-0.5) and (0.3,-1.0)},
                    edge from parent path={(\tikzparentnode\tikzparentanchor) |- (\tikzchildnode\tikzchildanchor)}]
                \node (nx) at (0,0) {\texttt{x}}
                    child {node {\texttt{y}}
                        child {node {} edge from parent [draw=none]}};
                \node[right of=nx] {\(\gets \text{access}\)};
                \end{tikzpicture}
            }
        \end{block}
    \end{minipage}
    ~\ ~\
    \begin{minipage}{0.28\textwidth}
        \begin{block}{}
            {\tiny
            \begin{lstlisting}[language=rust, basicstyle=\ttfamily\scriptsize]
let y = &*x;
(&*x).do_something();
            \end{lstlisting}
            }
            {\small
                \begin{tikzpicture}[
                    every node/.append style = {anchor = west},
                    grow via three points={one child at (0.3,-0.5) and two children at (0.3,-0.5) and (0.3,-1.0)},
                    edge from parent path={(\tikzparentnode\tikzparentanchor) |- (\tikzchildnode\tikzchildanchor)}]
                \node (nx) at (0,0) {\texttt{x}}
                    child {node(nrefx) {\texttt{\&*x}}}
                    child {node {\texttt{y}}};
                \node[right of=nrefx] {\(\gets \text{access}\)};
                \end{tikzpicture}
            }
        \end{block}
    \end{minipage}


    \begin{block}{}
        {\small
            \begin{tikzpicture}[
                every node/.append style = {anchor = west},
                grow via three points={one child at (0.3,-0.5) and two children at (0.3,-0.5) and (0.3,-1.0)},
                edge from parent path={(\tikzparentnode\tikzparentanchor) |- (\tikzchildnode\tikzchildanchor)}]
            \node (nx) at (0,0) {\texttt{?}}
                    child {node {\texttt{y}}};
            \node[right of=nx] {\(\gets \text{access}\)};
            \end{tikzpicture}
        }
    \end{block}
\end{frame}

\begin{frame}
    \frametitle{One pointer, \(2 \times 2\) kinds of accesses}
    \begin{block}{}
        \centering
        \begin{tikzpicture}[
            every node/.append style = {anchor = west},
            grow via three points={one child at (0.3,-0.5) and two children at (0.3,-0.5) and (0.3,-1.0)},
            edge from parent path={(\tikzparentnode\tikzparentanchor) |- (\tikzchildnode\tikzchildanchor)}]
            \node (root) at (0,0) {\texttt{?}}
                child {node {\texttt{?}}
                    child {node (nprevlast) {\texttt{?}}}
                    child {node (nxfirst) {\color{red}\texttt{x}}
                        child {node {\texttt{?}}}
                        child {node {\texttt{?}}
                            child {node (nxlast) {\texttt{?}}}}}
                    child[missing] {}
                    child[missing] {}
                    child[missing] {}
                    child {node (nnextfirst) {\texttt{?}}
                        child {node (nnextlast) {\texttt{?}}}}}
                ;
            \draw (2.5,0 |- root) node (vmark) {};
            \draw [decorate, decoration = {brace}] (vmark |- nxfirst) -- (vmark |- nxlast)
                node[midway] {~\textit{child} accesses for \color{red}\texttt{x}};
            \draw [decorate, decoration = {brace}] (vmark |- root) -- (vmark |- nprevlast)
                node[midway] {~\textit{foreign} accesses for \color{red}\texttt{x}};
            \draw [decorate, decoration = {brace}] (vmark |- nnextfirst) -- (vmark |- nnextlast)
                node[midway] {~\textit{foreign} accesses for \color{red}\texttt{x}};
        \end{tikzpicture}
        ~\\
        each \textit{read} or \textit{write}.
    \end{block}
    \begin{block}{Remark}
        Child accesses can be detected locally,
        so usually all unknown accesses are \textit{foreign} accesses.
    \end{block}
\end{frame}

\begin{frame}
    \frametitle{One tag, several pointers}
    \begin{block}{}
        \centering
        \begin{tikzpicture}[
            every node/.append style = {anchor = west},
            grow via three points={one child at (0.3,-0.5) and two children at (0.3,-0.5) and (0.3,-1.0)},
            edge from parent path={(\tikzparentnode\tikzparentanchor) |- (\tikzchildnode\tikzchildanchor)}]
        \node (nx1) at (0,0) {\texttt{x}}
                child {node {\texttt{y}}};
        \node (nx2) at (3,0) {\texttt{x}}
                child {node {\texttt{?}}
                    child {node {\texttt{y}}}};
        \node (nx2) at (6.5,0) {\texttt{x}}
                child {node {\texttt{?}}
                    child {node {\texttt{?}}
                        child {node {\texttt{y}}}}};
        \end{tikzpicture}
        ~\\
        \(\texttt{x} \to \texttt{y}\): foreign
        \(\qquad\)
        \(\texttt{y} \to \texttt{x}\): child;
    \end{block}
    \begin{block}{}
        \centering
        \begin{tikzpicture}[
            every node/.append style = {anchor = west},
            grow via three points={one child at (0.3,-0.5) and two children at (0.3,-0.5) and (0.3,-1.0)},
            edge from parent path={(\tikzparentnode\tikzparentanchor) |- (\tikzchildnode\tikzchildanchor)}]
        \node (nx1) at (0,0) {\texttt{x = y}};
        \node (nx2) at (3,0) {\texttt{x}}
                child {node {\texttt{y}}};
        \end{tikzpicture}
        ~\\
        \(\texttt{x} \to \texttt{y}\): child/foreign
        \(\qquad\)
        \(\texttt{y} \to \texttt{x}\): child;
    \end{block}
    \begin{block}{Later}
        This will be useful for raw pointers.
    \end{block}
\end{frame}

\begin{frame}
    \frametitle{Updates of permissions: general mechanics}
    \begin{block}{}
        \begin{align*}
                && \texttt{... | RW | RW | RW | ~~ | R~ | R~ | RW | ...} \\
              + && \texttt{~~~~~~~~~~<-~ foreign read ->~~~~~~~~~~~~~~~} \\
            \to && \texttt{... | RW | R~ | R~ | ~~ | R~ | R~ | RW | ...} \\
            UB: && \text{if the pointer is not allowed to lose write permissions}
        \end{align*}
    \end{block}
    \begin{block}{}
        \begin{align*}
                && \texttt{... | RW | RW | RW | ~~ | R~ | R~ | RW | ...} \\
              + && \texttt{~~~~~~~~~~<-~ child read ~~->~~~~~~~~~~~~~~~} \\
            \to && \texttt{... | RW | RW | RW | ~~ | R~ | R~ | RW | ...} \\
            UB: && \text{if the pointer does not currently have read permissions}
        \end{align*}
    \end{block}
\end{frame}

