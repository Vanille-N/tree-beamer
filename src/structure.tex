\section{Tree Structure}

% VERB: Are you also keeping track of the well-formedness of the tree somehow? 
% VERB: Is it an assumption or do you throw an error if you somehow detect a cycle?

\begin{frame}[fragile, t]
    \frametitle{When are pointers different ?}
    LLVM and Rust specifications: ``other references/pointers''

    Suggests that two pointers to the same data are ``different''.\\~\\

    \begin{onlyenv}<2>
        A pointer in our semantics is:
        \begin{lstlisting}[language=rust]
struct Pointer {
    address: usize,
    size: usize,
    tag: usize, // <- added specifically for TB
}
        \end{lstlisting}~\\

        Two pointers to the same data are not equal for TB
        if they have different tags.
    \end{onlyenv}
\end{frame}

\begin{frame}[fragile]
    \frametitle{A Tree of pointers}
    \begin{minipage}{0.27\textwidth}
        \begin{block}{}
            \begin{tikzpicture}[
                every node/.append style = {anchor = west},
                grow via three points={one child at (0.5,-0.8) and two children at (0.5,-0.8) and (0.5,-1.6)},
                edge from parent path={(\tikzparentnode\tikzparentanchor) |- (\tikzchildnode\tikzchildanchor)}]
            \node at (0,0) {\texttt{x}}
                child {node {\texttt{y1}}
                    child {node {\texttt{y2}}}}
                child[missing] {}
                child {node {\texttt{z1}, \texttt{z2}}}
                child {node {\texttt{w1}}
                    child {node {\texttt{w2}}
                        child {node {\texttt{w3}}}}};
            \end{tikzpicture}
        \end{block}
    \end{minipage}
    ~\ ~\
    \begin{minipage}{0.65\textwidth}
        \begin{block}{}
            \begin{lstlisting}[language=rust]
let x = &mut 0u64; // initial borrow

let y1 = &mut *x; // mutable reborrow
let y2 = &*y1; // shared reborrow

let z1 = &*x;
let z2 = z1 as *const u64; // cast

foo(x); // implicit mutable reborrow
fn foo(w2: &mut u64) {
    let w3 = &*w2;
}

            \end{lstlisting}
        \end{block}
    \end{minipage}
\end{frame}

\begin{frame}
    \frametitle{What's in the tree ?}
    Each pointer is given a tag~\\~\\

    Tree Borrows tracks:
    \begin{itemize}
        \item \textbf{permission}: per tag, per location;
        \item \textbf{hierarchy} between tags;
        \item accesses are done through a tag:
            \begin{itemize}
                \item \textbf{require permissions} of the tag\\
                    (UB if the permissions are insufficient)
                \item \textbf{update permissions} of other tags\\
                    (UB if the modification is forbidden)
            \end{itemize}
    \end{itemize}
\end{frame}

\begin{frame}
    \frametitle{One pointer, \(2 \times 2\) kinds of accesses}
    \begin{block}{}
        \centering
        \begin{tikzpicture}[
            every node/.append style = {anchor = west},
            grow via three points={one child at (0.3,-0.5) and two children at (0.3,-0.5) and (0.3,-1.0)},
            edge from parent path={(\tikzparentnode\tikzparentanchor) |- (\tikzchildnode\tikzchildanchor)}]
            \node (root) at (0,0) {\texttt{?}}
                child {node (nfather) {\texttt{?}}
                    child {node (nxfirst) {\color{red}\texttt{x}}
                        child {node {\texttt{?}}}
                        child {node {\texttt{?}}
                            child {node (nxlast) {\texttt{?}}}}}
                    child[missing] {}
                    child[missing] {}
                    child[missing] {}
                    child {node (nnextfirst) {\texttt{?}}}
                    child {node {\texttt{?}}
                        child {node (nnextlast) {\texttt{?}}}}}
                ;
            \draw (2.5,0 |- root) node (vmark) {};
            \draw [decorate, decoration = {brace}] (vmark |- nxfirst) -- (vmark |- nxlast)
                node[midway] {~\textit{child} accesses for \color{red}\texttt{x}};
            \draw [decorate, decoration = {brace}] (vmark |- root) -- (vmark |- nfather)
                node[midway] {~\textit{foreign} accesses for \color{red}\texttt{x}};
            \draw [decorate, decoration = {brace}] (vmark |- nnextfirst) -- (vmark |- nnextlast)
                node[midway] {~\textit{foreign} accesses for \color{red}\texttt{x}};
        \end{tikzpicture}
        ~\\
        each \textit{read} or \textit{write}.
    \end{block}
    \begin{block}{``pointers based on...''}
        LLVM specification: ``pointer \texttt{y} is based on pointer \texttt{x}''\\
        \(\simeq\) TB: ``\texttt{y} is a child of \texttt{x}''.
    \end{block}
\end{frame}

\begin{frame}[fragile]
    \frametitle{Kinds of accesses: examples}
    \begin{block}{}
        \begin{lstlisting}[language=rust]
let x = &mut ...;
let y = &mut *x;

*x = 1; // Write access; foreign for y; child for x.
let _ = *y; // Read access; child for y; child for x.
        \end{lstlisting}
    \end{block}
    \begin{block}{}
        \centering
        \begin{tikzpicture}[
            every node/.append style = {anchor = west},
            grow via three points={one child at (0.3,-0.5) and two children at (0.3,-0.5) and (0.3,-1.0)},
            edge from parent path={(\tikzparentnode\tikzparentanchor) |- (\tikzchildnode\tikzchildanchor)}]
            \node {...}
                child {node {\texttt{x}}
                    child {node {\texttt{y}}}};
        \end{tikzpicture}
    \end{block}
\end{frame}

\begin{frame}
    \frametitle{Summary}
    \begin{itemize}
        \item pointers identified by a tag;
        \item tags are stored in a tree structure;
            \begin{itemize}
                \item reborrows create fresh tags,
                \item new tag is a child of the reborrowed tag
            \end{itemize}
        \item each tag has per-location permissions;
            \begin{itemize}
                \item permissions allow or reject \textit{child accesses}\\
                    (done through child tags)
                \item permissions evolve in response to \textit{foreign accesses}\\
                    (done through non-child tags).
            \end{itemize}
    \end{itemize}
\end{frame}
